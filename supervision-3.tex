\documentclass{supervision}
\usepackage{course}
\usepackage{code}

\Supervision{3}

\begin{document}
\begin{questions}
  \question A C programmer is working with a little-endian machine with 8 bits
    in a byte and 4 bytes in a word. The compiler supports unaligned access and
    uses 1, 2 and 4 bytes to store \lstinline|char|, \lstinline|short| and
    \lstinline|int| respectively. The programmer writes the following
    definitions (code as written below) to access values in main memory (in the
    table below):

    \begin{cpp}[gobble=6]
      Address | Byte offset
      ....... | 00 01 02 03
      ________|_____________
      0x04 .. | 10 00 00 00
      0x08 .. | 61 72 62 33
      0x0c .. | 33 00 00 00
      0x10 .. | 78 0c 00 00
      0x14 .. | 08 00 00 00
      0x18 .. | 01 00 4c 03
      0x1c .. | 18 00 00 00

      int **i=(int **)0x04;
      short **pps=(short **)0x1c;
      struct i2c {int i; char *c;}*p=(struct i2c*)0x10;
    \end{cpp}

    \begin{parts}
      \part[8] Write down the values for the following C expressions:
        \begin{cpp}[gobble=8]
          **i
          p->c[2]
          &(*pps)[1]
          ++p->i
        \end{cpp}

      \part[4] Explain why the code shown below, when executed, will print the
        value $420$.
        \begin{cpp}]gobble=8]
          #include<stdio.h>

          #define init_employee(X,Y) {(X),(Y),wage_emp}
          typedef struct Employee Em;
          struct Employee {
          int hours,salary;
          int (*wage)(Em*);
          };

          int wage_emp(Em *ths) {
          return ths->hours*ths->salary;
          }

          #define init_manager(X,Y,Z) {(X),(Y),wage_man,(Z)}
          typedef struct Manager Mn;
          struct Manager {
          int hours,salary;
          int (*wage)(Mn*);
          int bonus;
          };
          int wage_man(Mn *ths){
          return ths->hours*ths->salary+ths->bonus;
          }

          int main(void) {
          Mn m = init_manager(40,10,20);
          Em *e= (Em *) &m;
          printf("%d\n",e->wage(e));
          return 0;
          }
        \end{cpp}

      \part[8] Rewrite the C code shown in \emph{part (b)} using C++ primitives
        and give four reasons why your C++ solution is better than the C one.
    \end{parts}

  \question Rewrite your \lstinline|Matrix| class to use templates to determine
    the dimensions. You should implement functions for $+$,$-$ and $*$ making
    sure that they return matrices of the correct dimensions. Give typedef's to
    define \lstinline|Matrix| and \lstinline|Vector| as before.

  \question C++11 added rvalue references to help programmers avoid unnecessary
    copying of objects. Find out what this means and write an explanation
    including an example which demonstrates that copying has been avoided. You
    might like to start here: \url{http://www.artima.com/cppsource/rvalue.html}

    \emph{Note: that for g++ you need to pass the switch -std=c++11 to enable
    C++11 features}

\end{questions}
\end{document}
