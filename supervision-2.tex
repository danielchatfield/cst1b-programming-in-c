\documentclass{supervision}
\usepackage{course}
\usepackage{code}

\Supervision{2}

% Please hand in a printed copy of your work to student admin IN THE CL by
% 10am on Wednesday 3rd December (i.e. before your last set of lectures).
% In addition, please send me a zip file of your code via email in case we
% want to try anything you've written.
%
% I propose we have a supervision at the following times on Thursday 4th
% December in Robinson:
%
% 0930 Priyesh (pp384), Will (wes24)
% 1030 Alex (ajb300), Jack (jhw52)
% 1130 James (jdb83), Daniel (dc584)

\begin{document}
\begin{questions}
    \question
    Write an implementation of a class \lstinline|LinkList| which stores zero or more positive integers internally as a linked list on the heap. The class should provide appropriate constructors and destructors and a method \lstinline|pop()| to remove items from the head of the list. The method \lstinline|pop()| should return \lstinline|-1| if there are no remaining items. Your implementation should override the copy constructor and assignment operator to copy the linked-list structure between class instances. You might like to test your implementation with the following:

    \begin{cpp}
    int main() {
        int test[] = {1,2,3,4,5};
        LinkList l1(test+1,4), l2(test,5);
        LinkList l3=l2, l4;
        l4=l1;
        printf("%d %d %d\n",l1.pop(),l3.pop(),l4.pop());
        return 0;
    }
    \end{cpp}

    \textit{Hint: heap allocation \& deallocation should occur exactly once!}

    \begin{solution}
        \cppfile{q1.cpp}
    \end{solution}

    \question
    If a function $f$ has a static instance of a class as a local variable, when might the class constructor be called?
    \begin{solution}

    \end{solution}

    \question
    Write a class \lstinline|Matrix| which allows a programmer to define $2 \times 2$ matrices. Overload the common operators (e.g. $+$, $-$, $*$, and $/$). Can you easily extend your design to matrices of arbitrary size?
    \begin{solution}
        \cppfile{q3.h}
        \noindent\rule{\linewidth}{0.4pt}
        \cppfile{q3.cpp}
        This cannot easily be extended to matrices of arbitrary size as the arithmetic operations only work for matrices of certain sizes. For operations that are valid (e.g. multiplication of two $3 \times 3$ matrices) the implementation would have to be modified to store the elements as a 2D array.
    \end{solution}

    \question
    Write a class \lstinline|Vector| which allows a programmer to define a vector of length two. Modify your \lstinline|Matrix| and \lstinline|Vector| classes so that they interoperate correctly (e.g. \lstinline|v2 = m*v1| should work as expected).
    \begin{solution}
        \cppfile{q4.cpp}
    \end{solution}

    \question
    Why should destructors in an abstract class almost always be declared \lstinline|virtual|?
    \begin{solution}
        Otherwise when a class that inherits from it is casted back to the abstract class then the destructor defined in the abstract class will be called which may cause a memory leak if the inheriting class uses more memory.
    \end{solution}

    \question
    Provide an implementation for: \lstinline|template<class T> T Stack<T>::pop();| and \lstinline|template<class T> Stack<T>::~Stack();| as declared in the slides for lecture 7.
    \begin{solution}
        \cppfile{q9.cpp}
        I initially wanted to do this by creating a destructor for the \lstinline|Item| class but given it wasn't in the header file I figured we were meant to do it this way.
    \end{solution}

    \question
    Provide an implementation for: \lstinline|Stack(const Stack& s);| and \\ \lstinline|Stack& operator=(const Stack& s);| as declared in the slides for lecture 7.
    \begin{solution}

    \end{solution}

    \question
    Using meta programming, write a templated class \lstinline|Prime|, which evaluates whether a literal integer constant (e.g. $7$) is prime or not at compile time.

    \question
    How can you be sure that your implementation of class \lstinline|Prime| has been evaluated at compile time?


\end{questions}
\end{document}
